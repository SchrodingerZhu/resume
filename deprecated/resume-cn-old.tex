\documentclass[margin, line]{res}
% This is my resume
% Chinese translation
% by ice1000

\usepackage{color, graphicx, hyperref, CJKutf8}

\oddsidemargin  -0.3in
\evensidemargin -0.3in
\resumewidth    = 1.5in
\itemsep        = 0.1in
\parsep         = 0.0in
\sectionskip    = 0.1in
\sectionwidth   = 1.2in

\newenvironment{list1}{
  \begin{list}
    {\ding{113}}{
      \setlength{\itemsep}{0in}
      \setlength{\parsep}{0in}
      \setlength{\parskip}{0in}
      \setlength{\topsep}{0in}
      \setlength{\partopsep}{0in}
      \setlength{\leftmargin}{0in}
    }
    {\end{list}}
}

\begin{document}
\name{
  {\sc 张寅森 (ice1000)}
  (ice1000kotlin@foxmail.com)
  \vspace*{.1in}
}
\begin{CJK}{UTF8}{gbsn}
  \begin{resume}
    % platform I am familiar with
    \section{\sc 熟悉的平台}
    Java SE \& Android \\
    Flutter \& Fuchsia \\
    JetBrains MetaProgrammingSystem
    
    % my education experience
    \section{\sc 教育经历}
    \textbf{成都外国语学校}, 2015.9 至今
    
    % my GH / SO account
    \section{\sc 社区}
    博客: \url{http://ice1000.org/} \\
    \url{https://github.com/ice1000} \\
    \url{http://stackoverflow.com/users/7083401/ice1000} \\
    在 GitHub 和 StackOverflow 上长期保持活跃,保持向许多著名开源项目提交 issue (如 Ruby/Tk)。
    
    % compiler theories
    \section{\sc 编译原理}
    \url{https://github.com/lice-lang} \\
    高度可扩展的解释型程序语言(支持 call-by-need/name/value),包含工具链、语言参考以及一个简单的 Haskell 实现。

    % language Java
    \section{\sc Java}
    \textbf{2 年开发经验} \\
    熟悉 Android 开发和 JNI 开发. \\
    \url{https://github.com/ice1000/algo4j} :
    一个 JNI 实现的算法 (及数据结构) 库,有严格测试,很多情况下比标准库快。
    
    % language Kotlin
    \section{\sc Kotlin}
    \textbf{2 年开发经验} \\
    开发了
    \url{https://github.com/icela/FriceEngine},
    和另外
    \textbf{三个} 项目,被
    \href{https://kotlin.link/?q=ice} {Awesome Kotlin}
    收录。
    
    % language Haskell
    \section{\sc Haskell}
    在
    \href{https://www.codewars.com/users/ice1000} {CodeWars}
    上,以 Haskell 为主,达到
    \textbf{1 kyu}
    ,排名 \#73。 \\
    熟悉 Parser Combinator 的使用,开发了
    \url{https://github.com/ice1000/Kt2Dart} \\
    一个解析 Kotlin 代码并将其翻译到 Dart 的翻译器。

    % my academic interests
    \section{\sc 学术兴趣}
    程序设计语言理论 (PLT) \\
    面向语言编程 (LOP, MPS) \\
    函数式编程 (Haskell, Scala)

    % my industrial interests
    \section{\sc 工业兴趣}
    移动端开发 (Android, Fuchsia) \\
    编译器相关工作 (CFA, 优化, Parser)

    \vspace*{-.2in}

  \end{resume}
\end{CJK}
\end{document}
