% !TEX program = xelatex
% This is my resume
% Chinese translation
% by ice1000

\documentclass{resume}

\usepackage{lastpage}
\usepackage{fancyhdr}
\usepackage{linespacing_fix} % disable extra space before next section
\usepackage[fallback]{xeCJK}

%% \setmainfont[]{SimSun}
%% \setCJKfallbackfamilyfont{rm}{HAN NOM B}
% \setCJKmainfont{Source Han Serif SC Regular}
%% \renewcommand{\thepage}{\Chinese{page}}

\begin{document}
% \pagestyle{fancy}
% \fancyhf{}
\renewcommand\headrulewidth{0pt}
% \cfoot{\thepage\ of \pageref{LastPage}}

\name{张寅森}

\basicInfo{
  \email{ice1000kotlin@foxmail.com} \textperiodcentered\
  \phone{(+86) 180-8192-5082} \textperiodcentered\
  \github[ice1000]{https://github.com/ice1000} \textperiodcentered\
  \homepage{https://ice1000.org}
  % \linkedin[user]{https://www.linkedin.com/in/user}
}

\section{教育经历}
\datedsubsection{\textbf{宾夕法尼亚州州立大学}, 美国}{2018.8 -- 现在}
  专业:计算机科学,预计毕业日期: 2022 年 6 月, GPA 3.25/4.00

\section{工作经历}
\datedsubsection{\textbf{前海源伞}, 深圳, 中国}{2018.2 -- 2018.7}
\role{静态分析}{编译器前端,IDE 插件开发实习}
\begin{itemize}
  \item 负责 pinpoint 分析器的 IntelliJ/CLion/Eclipse 工具集成,协助开发 SonarQube 插件
  \item 编写了一个多线程的跨 Java/Kotlin 的源代码索引工具,索引 Hadoop 仅需 4 分钟
  % \item 学到了很多 Linux 编程和 Clang/LLVM 相关的知识
\end{itemize}

\datedsubsection{\textbf{PingCAP}, 远程}{2018.8 -- 2019.8}
\role{分布式存储系统}{TiKV 实习 - Ecosystem 小组}
\begin{itemize}
  \item 改进各种 TiKV 的外部依赖库,如优化
    \href{https://docs.rs/crate/grpcio} {grpcio}
    的内存性能,增加
    \href{https://docs.rs/crate/procinfo} {procinfo}
    的功能
  \item 协助迁移 TiKV 及其 Raft 实现所使用的 Protocal-Buffer 库
  % \item 学到了很多 Rust 编程、分布式和数据库相关的知识
\end{itemize}

\section{个人项目}
% \datedsubsection{\textbf{FriceEngine}}{\url{https://icela.github.io/}}
% % \role{Kotlin, C\#, Racket, Ruby}{发起者和 Kotlin/C\#/Ruby 版本的主维护者}
% 一个跨语言、易使用的游戏引擎系列
% \begin{itemize}
%   % \item 易于使用,仅需实现生命周期方法,然后调用非常方便的 API 。
%   % \item 易于安装,基于各语言内置的 GUI 框架。
%   \item 基于生命周期方法和工具 API ,并基于各语言内置的 GUI 框架,易于安装。
%   \item 使用 GitHub 的 issue 和 milestone 功能作为任务和版本管理工具,有完善的改动记录和文档。
%   \item 提供一个基于 JavaFX 的拖拽式可视化设计器,所见即所得,可以生成各种语言的代码。
% \end{itemize}

% \datedsubsection{\textbf{DevKt}}{\url{https://github.com/ice1000/dev-kt}}
% 跨平台轻量级代码编辑器兼 Kotlin IDE
% \begin{itemize}
  % \item 内置 Java/Kotlin 的高亮、补全,其他语言可以借助插件(可移植自 JetBrains IDE)做到同样的支持。 \\
    % 对 Kotlin 有额外的编译运行支持。
  % \item 架构灵活,编辑器上层逻辑和 UI 框架彻底解藕,易于向其他 UI 框架移植。
  % \item 提供细粒度的高亮颜色和快捷键设置,设置可以热更新。
% \end{itemize}

\datedsubsection{\textbf{Voile}}{\url{https://github.com/owo-lang/voile-rs}}
实验型依赖类型编程语言,支持 Row-polymorphism
\begin{itemize}
  \item 支持依赖类型,值的自动推导(元变量)以及非依赖的 row-polymorphism 的 Record 和 Variant 类型。
  \item 借助 Rust 语言的生态系统实现命令行解析、代码解析和支持命令补全的交互式解释器。
\end{itemize}

% \datedsubsection{\textbf{minitt-rs}}{\url{https://github.com/owo-lang/minitt-rs}}
% 实验型依赖类型编程语言,基于 \href{http://www.cse.chalmers.se/~bengt/papers/GKminiTT.pdf} {Mini-TT}
% \begin{itemize}
  % \item 支持$\Pi$类型(返回类型依赖参数)、$\Sigma$类型(元组成员的类型依赖其他成员的值)、和类型和模式匹配。
  % \item 利用类型系统中和类型是一等公民的优势,扩展了和类型的子类型关系,同时保持支持递归类型。
  % \item 借助 Rust 语言的生态系统实现命令行解析、代码解析和支持命令补全的交互式解释器。
% \end{itemize}

% \datedsubsection{\textbf{Lice 语言}}{\url{https://github.com/lice-lang/lice}}
% \role{Kotlin, Java}{发起者和主维护者}
% 高度可扩展的解释型程序语言,运行在 JVM 上
% \begin{itemize}
  % \item 支持 lambda 和 惰性求值 (call by need) / 正则序求值 (call by name) / 严格求值 (call by value)。
  % \item 运行速度约为 Java (Hotspot 8u151) 的二十分之一,提供 Java 交互和脚本引擎支持。
  % \item 提供支持 GHCi 风格的代码补全、彩色输出的命令行交互式解释器和支持基于语义的高亮、补全、重命名、 \\
    % 定义跳转、求值替换、快速修复等功能的 JetBrains IDE 插件。
% \end{itemize}

% \datedsubsection{\textbf{Julia-IntelliJ}}{\url{https://github.com/JuliaEditorSupport/julia-intellij}}
% JetBrains IDE 的 Julia 插件
% \begin{itemize}
%   \item 支持基于语义的高亮、错误检查、快速修复、定义跳转、参数提示、补全、针对 Unicode 字符的特殊输入。
%   \item 集成 Markdown 插件高亮文档字符串,提供 REPL、SciView(展示 Plot 库的输出) 支持。
% \end{itemize}

\datedsubsection{\textbf{IntelliJ Pest}}{\url{https://github.com/pest-parser/intellij-pest}}
JetBrains IDE 的 Pest (Rust 编写的 parser generator) 插件
\begin{itemize}
  \item 支持基于语义的高亮、错误检查、定义跳转、变量补全、提取定义、内联定义以及与 Rust 插件集成。
  \item 提供实时高亮功能——可根据语法定义为用户代码动态提供高亮以测试语法定义文件,并支持导出 HTML。
\end{itemize}

% Reference Test
%\datedsubsection{\textbf{Paper Title\cite{zaharia2012resilient}}}{May. 2015}
%An xxx optimized for xxx\cite{verma2015large}
%\begin{itemize}
%  \item main contribution
%\end{itemize}

%% \section{\faHeartO\ 成就}
%% \datedline{}{Aug. 2017}

\section{技能}
\begin{itemize}[parsep=0.25ex]
  \item \textbf{编程语言}:
    \textbf{泛语言}(编程不受特定语言限制),
    且尤其熟悉 Java/Kotlin/Rust/C\#/Agda/Haskell,
    较为熟悉 Dart/C/C++/F\#/F$\star$/Idris/Perl (均不分先后)

  % compiler theories
  \item \textbf{编译原理}:
    理解各种程序表示,例如控制流图或者 A-Normal Form

  % language Kotlin
  \item \textbf{Kotlin/Java}:
    \textbf{3 年开发经验},
    % \textbf{4} 个项目被
    % \href{https://kotlin.link/?q=ice} {Awesome Kotlin}
    % 收录,
    熟悉 JNI、Gradle、Swing,
    理解 Contract DSL 和 Kotin coroutines,
    有使用 Kotlin 编译器分析代码的经验

  \item \textbf{类型系统}:
    掌握 Martin-L\"{o}f 类型论、余代数和 Cubical 类型论,
    正在学习同伦类型论;
    熟悉 Idris, Agda (\textbf{1 年}使用经验,开发组成员之一),
    F$\star$ 和一些 Coq
    \subitem 读过代码的项目: Agda, Idris, MiniAgda, Mini-TT, redtt, TOG

  \item \textbf{JetBrains MPS}:
    理解\textbf{面向语言编程}的概念和应用

  \item \textbf{IDE 工具开发}:
    \textbf{2 年开发经验},
    熟悉 IntelliJ 平台的基础设施和整体架构(开发了
      \href{https://plugins.jetbrains.com/plugin/10413}
           {Julia 插件}、
      \href{https://plugins.jetbrains.com/plugin/12176}
           {DTLC 插件},
      \href{https://plugins.jetbrains.com/plugin/12046}
           {Pest 插件}和很多其他插件),
    同时了解 Eclipse/SonarQube 的插件开发

  % platforms I am familiar with  
  \item \textbf{移动开发}:
    \textbf{2 年开发经验},
    Android (Java, Kotlin), Fuchsia (Flutter)

  \item \textbf{开发工具}:
    能适应任何编辑器/操作系统,平常在 Ubuntu 下使用 JetBrains IDE、Emacs,
    有使用 YouTrack、Jira、GitHub、BitBucket、Coding.net、Tower 等团队协作工具的经验
\end{itemize}

% \section{\faHeartO\ Honors and Awards}
% \datedline{\textit{\nth{1} Prize}, Award on xxx }{Jun. 2013}
% \datedline{Other awards}{2015}

\section{其他}
\begin{itemize}[parsep=0.25ex]
  \item \href{https://ice1000.org/lagda/}{Literate Agda 博客}(使用了我改进过的 Agda 文学编程模式)
  \item 一些个人页面链接(请使用支持超链接的 PDF 阅读器查看此项):
    \href{https://bintray.com/ice1000}{Bintray 主页}(用于发布 JVM 库)
  , \href{https://crates.io/users/ice1000}{Crates.io 主页}(用于发布 Rust 库)
  , \href{https://plugins.jetbrains.com/author/10a216dd-c558-4aaf-aa8a-723f431452fb}
    {IntelliJ 插件开发者主页}
  , \href{https://personal.psu.edu/yqz5714/}{科研计划}
  %% \item 我写的一些关于形式验证的书: \url{https://github.com/ice1000/Books}
  %% \item 开源贡献: \url{https://ice1000.org/opensource-contributions/} \\
  %%   向 \textit{Microsoft, JetBrains, Ruby, Dropbox, PingCAP, TiKV} 等组织,
  %%   \textit{agda, imgui, shields.io, intellij-solidity, IntelliJ-EmmyLua,
  %%     intellij-haskell} 等项目提交过功能性的 pull request
  \item 开源贡献: \url{https://ice1000.org/opensource-contributions/}
    % 为 \textsf{JuliaEditorSupport}、\textsf{agda}、\textsf{pest-parser}、\textsf{EmmyLua} 等组织的成员,
    % 除组织项目外还
    向 \textsf{agda, KaTeX, shields.io, raft-rs, grpc-rs, intellij-solidity,
      intellij-haskell, intellij-rust, rust-analyzer} 等项目贡献过代码
  \item StackOverflow: \url{https://tinyurl.com/y5cmw3dz},
    3000+ 声誉,同时也在\href{https://stackexchange.com/users/9532102/} {其他 StackExchange 子站}活跃
  \item 语言: English - 熟练 (托福 100),汉语 - 母语水平
  % https://raw.githubusercontent.com/ice1000/resume/master/resume-cn.pdf
  \item 获取此简历的最新版本: \url{https://tinyurl.com/ya4urea8}
  \item 获取此简历的完整版本(英语): \url{https://tinyurl.com/y2v59t36}
  \item 在
    \href{https://www.codewars.com/users/ice1000} {CodeWars}
    上,以 Haskell、Agda 和 Idris 为主,达到 \textbf{1 dan},全站排名 \#31
  \item 很喜欢交朋友
\end{itemize}

%% Reference
%\newpage
%\bibliographystyle{IEEETran}
%\bibliography{mycite}
\end{document}
